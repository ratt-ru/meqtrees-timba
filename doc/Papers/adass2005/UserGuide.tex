%APN3_PROCEEDINGS_FORM%%%%%%%%%%%%%%%%%%%%%%%%%%%%%%%%%%%%%%%%%%%%%%%
%
% UserGuide.tex -- ADASS VIII (98) ASP Conference Proceedings User Guide
% Updated by N. Manset for ADASS IX (99)
% Updated by D. Bohlender for ADASS XI (2001)
% Updated by H. Payne for ADASS XIII (2003)
% Updated by P. Shopbell for ADASS XIV (2004)
% Updated by C. Gabriel for ADASS XV (2005)
%
% Use this template to create your proceedings paper in LaTeX format
% by following the instructions given below.  Much of the input will
% be enclosed by braces (i.e., { }).  The percent sign, ``%'', denotes
% the start of a comment; text after it will be ignored by LaTeX.
% You might also notice in some of the examples below the use of ``\``
% after a period; this prevents LaTeX from interpreting the period as
% the end of a sentence and putting extra space after it.
%
% You should check your paper by processing it with LaTeX.  For
% details about how to run LaTeX as well as how to print out the User
% Guide, consult the README file.  (The User Guide is also available
% on-line via http://adass2002.stsci.edu/instructions/manuscripts/UserGuide/
% You should also consult the sample LaTeX papers, sample1.tex and
% sample2.tex, for examples of including figures, html links, special
% symbols, and other advanced features.
%
% If you do not have access to the LaTeX software or a laser printer
% at your site, you can still prepare your paper following the
% instructions in the User Guide.  In such cases, the editors will
% process the file and make any necessary editorial adjustments.
%
%%%%%%%%%%%%%%%%%%%%%%%%%%%%%%%%%%%%%%%%%%%%%%%%%%%%%%%%%%%%%%%%%%%%%%%%
%

\documentclass[11pt,twoside]{article}
\usepackage{adassconf}

% Some definitions I use in these instructions.
% these were lifted from the original aspconf.tex file
% YOU SHOULD NOT USE YOUR OWN DEFINITIONS
\def\emphasize#1{{\sl#1\/}}
\def\arg#1{{\it#1\/}}
\let\prog=\arg

\def\edcomment#1{\iffalse\marginpar{\raggedright\sl#1\/}\else\relax\fi}
\marginparwidth 1.25in
\marginparsep .125in
\marginparpush .25in
\reversemarginpar

\begin{document}

%-----------------------------------------------------------------------
%                           Paper ID Code
%-----------------------------------------------------------------------
% Enter the proper paper identification code.  The ID code for your
% paper is the session number associated with your presentation as
% published in the official conference proceedings.  You can           
% find this number locating your abstract in the printed proceedings
% that you received at the meeting or on-line at the conference web
% site; the ID code is the letter/number sequence proceeding the title 
% of your presentation. 
%
% This will not appear in your paper; however, it allows different
% papers in the proceedings to cross-reference each other.  Note that
% you should only have one \paperID, and it should not include a
% trailing period.
%
% EXAMPLE: \paperID{O0-1}
% EXAMPLE: \paperID{P7-7}
%

\paperID{O4-1} % bogus id since we need one

%-----------------------------------------------------------------------
%                           Paper Title
%-----------------------------------------------------------------------
% Enter the title of the paper.
%
% EXAMPLE: \title{A Breakthrough in Astronomical Software Development}
%
% If your title is so long as to fill the page header when you print it,
% then please supply a short form as a \titlemark.
%
% EXAMPLE:
%  \title{Rapid Development for Distributed Computing, with Implications
%         for the Virtual Observatory}
%  \titlemark{Rapid Development for Distributed Computing}
%

\title{Astronomical Data Analysis Software \& Systems XV:
   Instructions for Authors Using \LaTeX\ Markup}
\titlemark{Instructions for Authors}

%-----------------------------------------------------------------------
%		          Authors of Paper
%-----------------------------------------------------------------------
% Enter the authors followed by their affiliations.  The \author and
% \affil commands may appear multiple times as necessary (see example
% below).  List each author by giving the first name or initials first
% followed by the last name.  Authors with the same affiliations
% should grouped together. 
%
% EXAMPLE: \author{Raymond Plante, Doug Roberts, 
%                  R.\ M.\ Crutcher\altaffilmark{1}}
%          \affil{National Center for Supercomputing Applications, 
%                 University of Illinois Urbana-Champaign, Urbana, IL
%                 61801}
%          \author{Tom Troland}
%          \affil{University of Kentucky}
%
%          \altaffiltext{1}{Astronomy Department, UIUC}
%
% In this example, the first three authors, "Plante", "Roberts", and
% "Crutcher" are affiliated with "NCSA".  "Crutcher" has an alternate 
% affiliation with the "Astronomy Department".  The fourth author,
% "Troland", is affiliated with "University of Kentucky"

%\author{David Bohlender}
%\affil{Canadian Astronomy Data Centre, Herzberg Institute of Astrophysics,
%5071 West Saanich Road, Victoria, BC, Canada V9E 2E7}

%-----------------------------------------------------------------------
%                        Contact Information
%-----------------------------------------------------------------------
% This information will not appear in the paper but will be used by
% the editors in case you need to be contacted concerning your
% submission.  Enter your name as the contact along with your email
% address.
%
% EXAMPLE:  \contact{Dennis Crabtree}
%	    \email{Dennis.Crabtree@hia.nrc.ca}
%

\contact{Carlos Gabriel}
\email{carlos.gabriel@esa.int}

%-----------------------------------------------------------------------
%                     Author Index Specification
%-----------------------------------------------------------------------
% Specify how each author name should appear in the author index.  The
% \paindex{ } should be used to indicate the primary author, and the
% \aindex for all other co-authors.  You MUST use the following
% syntax:
%
% SYNTAX:  \aindex{LASTNAME, F. M.}
%
% where F is the first initial and M is the second initial (if
% used).  This guarantees that authors that appear in multiple papers
% will appear only once in the author index.
%
% EXAMPLE: \paindex{Crabtree, D.}
%	   \aindex{Manset, N.}        
%	   \aindex{Veillet, C.}        
%
% NOTE: this information is also used to build the author list that
% appears in the table of contents.  Authors will be listed in the order
% of the \paindex and \aindex commmands.
%

\paindex{Gabriel, C.}

%-----------------------------------------------------------------------
%                     Author list for page header
%-----------------------------------------------------------------------
% Please supply a list of author last names for the page header. in
% one of these formats:
%
% EXAMPLES:
% \authormark{LASTNAME}
% \authormark{LASTNAME1 \& LASTNAME2}
% \authormark{LASTNAME1, LASTNAME2, ... \& LASTNAMEn}
% \authormark{LASTNAME et al.}
%
% Use the "et al." form in the case of seven or more authors, or if
% the preferred form is too long to fit in the header.

\authormark{ }

%-----------------------------------------------------------------------
%                       Subject Index keywords
%-----------------------------------------------------------------------
% Enter up to 6 keywords describing your paper.  These will NOT be
% printed as part of your paper; however, they will be used to
% generate the subject index for the proceedings.  There is no
% standard list; however, you can consult the indices for past ADASS
% proceedings (http://adass.org/adass/proceedings/).
%
% EXAMPLE:  \keywords{visualization, astronomy: radio, parallel
%                     computing, AIPS++, Galactic Center}
%
% In this example, the author noticed that "radio astronomy" appeared
% in the ADASS VII Index as "astronomy" being the major keyword and
% "radio" as the minor keyword.

\keywords{publishing, publishing: electronic, guides: user, \LaTeX, 
          American Astronomical Society}

%-----------------------------------------------------------------------
%                              Abstract
%-----------------------------------------------------------------------
% Type abstract in the space below.  Consult the User Guide and Latex
% Information file for a list of supported macros (e.g. for typesetting 
% special symbols). Do not leave a blank line between \begin{abstract} 
% and the start of your text.

\begin{abstract}
This manual describes the procedure for creating a \LaTeX\ formatted
paper suitable for submission to the ADASS conference proceedings.
In particular, it describes how to employ special formatting techniques
for hypertext links, equations, tables, and figures.  Use the accompanying
\verb+template.tex+ file, and the examples here and in the sample documents
to prepare your paper.
\end{abstract}

%-----------------------------------------------------------------------
%                             Main Body
%-----------------------------------------------------------------------
% Place the text for the main body of the paper here.  You should use
% the \section command to label the various sections; use of
% \subsection is optional.  Significant words in section titles should
% be capitalized.  Sections and subsections will be numbered
% automatically.
%
% EXAMPLE:  \section{Introduction}
%           ...
%           \subsection{Our View of the World}
%           ...
%           \section{A New Approach}
%
% It is recommended that you look at the sample papers, sample1.tex
% and sample2.tex, for examples for formatting references, footnotes,
% figures, equations, html links, lists, and other special features.

\setcounter{footnote}{1}

\section{Introduction}

In order to ensure that papers received for publication from different
authors are consistent in format, style, and quality, authors are
required to type their manuscripts in \LaTeX\ format exactly according
to the following instructions.  The editors will modify the electronic
manuscripts as necessary to insure that they conform to these
standards.

Papers by invited speakers are allotted a maximum of ten (10) pages
in the Proceedings. Regular oral presentations, posters, and demos are
limited a maximum of four (4) pages each; BoF summaries are limited to one
(1) page.  Presenters of multiple papers will be permitted to submit a
manuscript for each for inclusion in the Proceedings, \emph{although
it is necessary that we receive a completed Publication Agreement
and Copyright Assignment form for each submission}.  Because of the
extremely high cost, color illustrations will not be published in the
paper version of the proceedings, but may be included in the electronic
version  of the proceedings. All authors should send black-and-white
figures, plus the color versions of their figures if they wish to.
Any figure sent in color version only will be reproduced in black and
white in the Proceedings book.

Authors must submit their manuscripts and associated figures
electronically by {\bf 31 October 2005}. See  
%\S\ref{submit}
\S 6
for details on using anonymous FTP to submit your files.  

\section{\LaTeX\ Markup Commands}

Authors should use the \verb+adassconf+ style file, and declare it as
a substyle to the standard \LaTeX\ \verb+article+ style.  A copy of
the style file, two sample papers, a template file, and further
instructions are available at the conference web site.  See the 
%\S\ref{submit}
\S 6
file for instructions for paper submissions.

You should use only those markup commands from \LaTeX\ plus the several
extensions provided by this style file.  Do \emphasize{not} define any
commands of your own for any reason (no \verb+\def+ or \verb+\newcommand+
statements).

\subsection{Preamble}

The first piece of markup in the manuscript must declare the
overall style of the document and invoke the \verb+adassconf+ package.
\begin{quote}
\verb+\documentclass[11pt,twoside]{article}+\\
\verb+\usepackage{adassconf}+
\end{quote}
The \verb+\documentclass+ command must appear first in any
\LaTeX\ file, and this one specifies the main style to be
the \LaTeX\ \verb+article+ style using eleven point fonts,
with modifications and additions for the \verb+adassconf+ package.

If you do not have access to \LaTeXe, but do have access to the 
old \LaTeX{} 2.09, then you can replace the \verb+\documentclass+ and
\verb+\usepackage+ lines with
\begin{quote}
\verb+\documentstyle[11pt,twoside,adassconf]{article}+
\end{quote}

The author must include a
\begin{quote}
\verb+\begin{document}+
\end{quote}
command to identify the beginning of the main portion of the manuscript.

\subsection{Paper ID Code}
\addtocounter{footnote}{-1}
Authors must include their proper paper identification code using the
\verb+\paperID+ command.  The ID code for your paper is the session number
associated with your presentation as published in the official conference 
program.  You can find this number locating your
abstract in the printed program that you received at the meeting or
via the on-line program;
% \htmladdnormallinkfoot{the on-line program}%
%{http://adass.stsci.edu/events/program/}; 
the ID code is the letter-number sequence proceeding the title of your
presentation.  

The paper ID code will not appear in your paper; however, it allows
different papers in the proceedings to cross-reference each other.
You should only have one \verb+\paperID+ per submission, and it
should not include a trailing period.

\subsection{Contact Information}

Authors should include a contact person and email address using the 
\verb+\contact+ and \verb+\email+ macros.  This information will not
appear in the paper but will be used by the editors in case you need to be
contacted concerning your submission.  Enter your name as the contact along
with your email address.

\subsection{Title, Byline, Abstract, etc.}

Title and author identification are by way of the standard \LaTeX\
commands \verb+\title+ and \verb+\author+.  An author's principal
affiliation is specified with a separate macro \verb+\affil+.  Each
\verb+\author+ command should be followed by a corresponding \verb+\affil+
(address).  If possible, authors should limit the number of \verb+\author+
commands by grouping authors by affiliations.
\begin{quote}
\verb+\title{+\arg{lucid text}\verb+}+\\
\verb+\author{+\arg{name(s)}\verb+}+\\
\verb+\affil{+\arg{address}\verb+}+
\end{quote}
The \verb+\affil+ command should be used to give the author's full postal
address.  The address will be broken over several lines automatically;
do \emphasize{not} use \LaTeX's \verb+\\+ command to indicate the line
breaks.  Please use mixed case text for \emphasize{all} these fields
rather than supplying all capitals; the style file will convert to upper
case as necessary.

The article must contain an abstract enclosed in an \verb+abstract+
environment:
\begin{quote}
\verb+\begin{abstract}+\\
\arg{abstract text}\\
\verb+\end{abstract}+
\end{quote}
Do not include the word ``Abstract'' in your text; it is inserted
automatically.  And do not leave a blank line between \verb+\begin{abstract}+
and the start of the text of the abstract.

\subsection{Author Index Specification}

Use the \verb+\paindex+ and \verb+\aindex+ macros to indicate how each
author name should appear in the author index.  The \verb+\paindex+ should
be used to indicate the primary (first) author, and the \verb+\aindex+
for all other co-authors.  You MUST use the following syntax: 
\begin{quote}
\verb+\aindex{+\arg{LASTNAME, F. M.}\verb+}+
\end{quote}
where {\it F} is the first initial and {\it M} is the second initial (if
used).  This guarantees that authors on multiple papers
will appear only once in the author index.

The \verb+\paindex+ and \verb+\aindex+ macros are used to build the author
list in the table of contents.  Therefore, the macros should appear in the 
order of the names in the \verb+\author+ macros.

\subsection{Page Headers}

An article title and author list can appear in headers on alternate
pages if the \verb+\twoside+ style option is used.  If the title of the
paper is too long to fit in the header, specify a shorter version using
the \verb+\titlemark+ command, as in:
\begin{quote}
\verb+\title{Rapid Development for Distributed Computing, with+\\
\verb+       Implications for the Virtual Observatory}+\\
\verb+\titlemark{Rapid Development for Distributed Computing}+
\end{quote}

By default, the author list is empty.  Use the \verb+\authormark+
command to supply a value in one of these formats:
\begin{quote}
\verb+\authormark{LASTNAME}+\\
\verb+\authormark{LASTNAME1 \& LASTNAME2}+\\
\verb+\authormark{LASTNAME1, LASTNAME2, ... \& LASTNAMEn}+\\
\verb+\authormark{LASTNAME et al.}+
\end{quote}
Use the "\verb+et al.+" form in the case of seven or more authors, or if
the preferred form is too long to fit in the header.

\subsection{Subject Index Keywords}

You must use the \verb+\keywords+ macro to enter up to 6 keywords
describing your submission.  These will NOT be printed as part of your
paper; however, they will be used to generate the subject index for the
proceedings.  There is no standard list; however, you can consult the
indices for \htmladdnormallinkfoot{past ADASS proceedings.}{http://adass.org/adass/proceedings/} 

\subsection{Sections}

The \LaTeX\ \verb+article+ environment supports two levels of sectioning.
(Actually, it supports more, but these are the relevant ones.)
\begin{quote}
\verb+\section{+\arg{heading}\verb+}+\\[.5ex]
\verb+\subsection{+\arg{heading}\verb+}+
\end{quote}
Please use mixed case text for the section heads:
\begin{quote}
``{\bf Conclusions and Future Work}'' instead of\\
``{\bf Conclusions and future work}''
\end{quote}
%Note that these commands delimit sections by marking the
%\emphasize{beginning} of each section; there are not separate commands
%to identify the \emphasize{ends}.

If one wishes to have an acknowledgments section, it should be
set off simply with the command
\begin{quote}
\verb+\acknowledgments+
\end{quote}

\subsection{Hypertext Links and URLs}

Since the proceedings will be published in both paper and electronic
form, you are encouraged to specify URLs to other relevant electronic
documents when appropriate.  Avoid links to personal files that may
disappear after a few months, making the links obsolete.  

The ADASS conference style provides several macros to ensure hypertext
links and URLs are formatted properly in each version.  The most
used commands are \verb+\htmladdnormallink+ and
\verb+\htmladdnormallinkfoot+.  These commands are analogous to the
\verb+<a +\arg{...}\verb+>+ tag in HTML, allowing you to link a piece
of text to a URL.  Both commands take two arguments: the link text and
the associated URL\@.  For example:
\begin{quote}
\verb+\htmladdnormallink{ADS}{http://adswww.harvard.edu/}+
\end{quote}
Using \verb+\htmladdnormallinkfoot+ will cause the URL to appear in
the printed copy of your paper as a footnote to that text (for example,
when one refers to the
\htmladdnormallinkfoot{ADS}{http://adswww.harvard.edu/}).  In the
on-line version, the text will be an actual HTML link to that URL.
\verb+\htmladdnormallink+ is just like \verb+\htmladdnormallinkfoot+
except that the URL does not appear in the printed version.  

If you wish to have the URL explicitly appear within the body of your
paper (rather than as a footnote) you can use the \verb+\makeURL+ or 
\verb+\htmladdURL+ command to format it:
\begin{quote}
\verb+\makeURL{+\arg{URL}\verb+}+\\
\verb+\htmladdURL{+\arg{URL}\verb+}+
\end{quote}
\verb+\htmladdURL+ will cause the URL to be a link to itself in the
on-line version; with \verb+\makeURL+, the URL will appear as plain
text.  

Note that it is not necessary to escape special characters like tilde
(\verb+~+) and underscore (\verb+_+) within your URLs when you enter
them as arguments to any of these four commands.  These special
characters will be properly formatted in both the on-line and printed
versions:  
\begin{quote}
\verb+\htmladdnormallink{my document}%+\\
\verb+{http://www.cfht.hawaii.edu/~crabtree/my_doc.html}+
\end{quote}

\subsection{Equations}

Displayed equations can be typeset in many ways using the standard
displayed math environments of \LaTeX;
these three are probably of greatest use:
\begin{quote}
\verb+\begin{displaymath}+\\
\verb+\end{displaymath}+\\[.5ex]
\verb+\begin{equation}+\\
\verb+\end{equation}+\\[.5ex]
\verb+\begin{eqnarray}+\\
\verb+\end{eqnarray}+
\end{quote}
The \verb+displaymath+ environment will break out a single, unnumbered
formula.  The equation will appear the same if it is set in an
\verb+equation+ environment, and it will be autonumbered by \LaTeX.
In order to set several formul\ae\ in which vertical alignment is
required, use the \verb+eqnarray+ environment.

\subsection{Tables}

Tables may be formatted using the \verb+deluxetable+ environment.  Details
about this table environment can be found in the AAS\TeX\ user guide
found online at the URL: {\it http://www.journals.uchicago.edu/AAS/AASTeX/}. 

\begin{quote}
\verb+\begin{deluxetable}{+\arg{cols}\verb+}+\\
\verb+\tablecaption{+\arg{text}\verb+}+\\
\verb+\tablehead{+\arg{column headings}\verb+}+\\
\verb+\colhead{+\arg{text}\verb+}+\\
\verb+\startdata+\\
{\it data}\\
\verb+\enddata+\\
\verb+\end{deluxetable}+
\end{quote}

The {\it cols} specifies the justification for each column.  One of
the letters `l', `c', or `r' is given for each column, indicating left,
center, or right justification.  The table width can be explicitly set
with the \verb+\tablewidth{+\arg{width}\verb+}+ command.

The font size of the table information
can be adjusted using the \verb+\small+, \verb+\footnotesize+, or
\verb+\scriptsize+ commands right after \verb+\begin{deluxetable}.+
(Reducing the size of the text will reduce the readability
of the table, however.)  {\em If you are using \LaTeXe, there is a bug in
the style that prevents this from working; you will have to use a
regular \verb+tabular+ environment.}
%AAS\TeX{} is incorporated into our style file.}

Tables may also appear in \verb+table+ environments, although the 
\verb+deluxetable+ environment is preferred.
\begin{quote}
\verb+\begin{table}+\\
\verb+\caption{+\arg{text}\verb+}+\\
\verb+\begin{tabular}{+\arg{cols}\verb+}+\\
\verb+\end{tabular}+\\
\verb+\end{table}+
\end{quote}
There should be only one table per environment. The \verb+table+
environment encloses not only the tabular material but also any title
(caption) or footnote information associated with the table. Tabular
information is typeset within \LaTeX's {\tt tabular} environment;
the \arg{cols} argument specifies the formatting for each column.
Tables and figures will be identified with arabic numerals, e.g.,
``Table 2.''; the identifying text, including the number, is generated
automatically by the \verb+\caption+ command.

The {\tt table} environment provides more control over column spacing
than the {\tt deluxetable} environment.  Instead of reducing the font
size when a table is too wide, it may be possible to use this control to
make it fit.  An example is given in the {\tt sample2.tex} document
(Table 3).

There is a \verb+\tableline+ command for use in {\tt tabular}
environments.  This command produces a single horizontal rule.
There should be two \verb+\tableline+'s above and one below between
the column headings, and one at the end of the table.  Authors should
not use additional \verb+\tablelines+, and are discouraged from using
vertical rules unless essential.

\subsection{Figures as EPS Files}
\label{figures}

Authors who can prepare computer graphics in Encapsulated PostScript
(EPS) format may use one of two additional markup commands to mark the
point of inclusion, both of which should be used inside a \LaTeX\
{\tt figure} environment.  If the DVI translator \prog{dvips} (by Tom
Rokicki) is available on your computer, it is also possible to prepare the
final copy with such figures in place.  

Before including the EPS figures in your text, be sure to rename the
EPS files to conform to the name of your \LaTeX\ file:
{\tt {\it O4-1\_1.eps}}, {\tt {\it O4-1\_2.eps}}, etc. You will use
these names in the markup commands for including EPS figures.
 
These markup commands are:
\begin{quote} 
\verb+\plotone{+\arg{file}\verb+}+\\
\verb+\plottwo{+\arg{file}\verb+}{+\arg{file}\verb+}+
\end{quote}
 
The \arg{file} argument is used to name the file(s) to be included.  The
\verb+\plotone+ command includes one figure that is scaled to the width of
the current text column; \verb+\plottwo+ inserts two figures side by side,
and the pair is scaled to fit the text width.  If one uses these
macros, the necessary vertical space is provided automatically.
 
\begin{quote}
\verb+\begin{figure}+\\  
\verb+\plotone{+\arg{O4-1\_1.eps}\verb+}+\\
\verb+\caption{+\arg{My EPS graphic.}\verb+}+\\
\verb+\end{figure}+
\end{quote}
or
\begin{quote}
\verb+\begin{figure}+\\
\verb+\plottwo{+\arg{O4-1\_1.eps}\verb+}{+\arg{O4-1\_2.eps}\verb+}+\\
\verb+\caption{+\arg{Two related graphics.}\verb+}+\\
\verb+\end{figure}+
\end{quote}
Please note that the caption will be centered under the {\it pair\/} of
graphics when \verb+\plottwo+ is used.  It is not possible to caption the two
plots individually with this package at this time.  As with tables, figures
will be identified with arabic numerals, e.g., ``Figure 1.''

The scaling of the EPS plot may be adjusted with the 
\verb+\epsscale{+\arg{scale}\verb+}+ command, i.e., \verb+\epsscale{0.8}+.
Specifying \verb+\epsscale{0.8}+ should make the figure 80\% as wide as
the text on the page.

The reason that EPS figures refuse to be positioned properly with
\verb+\plotone+ and \verb+\plottwo+ is usually a bad {\tt BoundingBox}
comment in the PostScript.  The bounding box is supposed to be the
smallest rectangle, with sides parallel to the edges of the paper, that
surrounds all of the marks on the page.  Extra white space can make a
figure off-centered or hard to scale.  If you can print the figure, the
problem can be fixed by editing the EPS file and changing the BoundingBox
comment, which contains four numbers:  lower-left $x$, lower-left $y$,
upper-right $x$, and upper-right $y$ coordinates, measured from the
lower-left hand corner of the paper in units of 72 per inch (0.35mm).  
If you use
a PostScript preview program like {\tt gs} or {\tt gv}, you can position
the cursor at the corners of the figure and read off the coordinates.

As a last resort, if further fussing with the positioning of plot on
the printed page is necessary, you can try using this command:

\begin{quote} \verb+\plotfiddle{+\arg{file}\verb+}{+\arg{vsize}\verb+}{+\arg{rot}\verb+}{+\arg{hsf}\verb+}{+\arg{vsf}\verb+}{+\arg{htrans}\verb+}{+\arg{vtrans}\verb+}+ \end{quote}

\begin{quote} 
\begin{tabular}{lp{3in}} 
\tt vsize & 
 vertical white space to allow for plot, any valid \LaTeX\ dimension\\ 
\tt rot & rotation angle, in degrees, counter-clockwise\\ 
\tt hsf & horiz scale factor, percent\\
\tt vsf & vert scale factor, percent\\ 
\tt htrans & horiz translation, in PS points 72/in (0.35mm)\\ 
\tt vtrans & vert translation, in PS points 72/in (0.35mm)\\
\end{tabular} 
\end{quote}

If you {\em can} produce EPS but you do {\em not} have \prog{dvips},
you can still put the \verb+\plotone+ or \verb+\plottwo+ commands in the
the appropriate places, but you will have to comment them out and put
in a \verb+\vspace{+\arg{dimen}\verb+}+ command  to open up the text.
The \prog{dvips} program is in the public domain and is available from
labrea.stanford.edu.

A special note to authors:  
%Color EPS files should be avoided if possible.
Since it is sometimes necessary to edit EPS files to make them 
printable, authors should try to avoid EPS files with lines longer than
1024 characters.

\subsection{Pasted in Illustrations}

Illustrations must be inserted in the text at the appropriate places, with
the relevant caption underneath each.  The finished pages are reduced by
10\% before printing.  Thus, illustrations will appear somewhat smaller
in print.

These illustrations should appear in {\tt figure} environments.
\begin{quote}
\verb+\begin{figure}+\\
\verb+\vspace{+\arg{dimen}\verb+}+\\
\verb+\caption{+\arg{text}\verb+}+\\
\verb+\end{figure}+
\end{quote}
There should be only one figure per environment.
Space for the figure is created with the \verb+\vspace+ command;
\arg{dimen} should be a valid \LaTeX\ dimension, e.g., ``\verb+2.5in+''
or ``\verb+7.1cm+''.

\edcomment{EPS takes care of all this.}

\subsubsection{Size of Illustrations}

The maximum width of an illustration is normally 13.4cm (5.25in)
so that it will fit within the width of the text area.  Of course an
illustration may be smaller if appropriate.  A large illustration may
be placed sideways (``landscape'') on the paper if necessary.

\subsubsection{Halftone Illustrations (Photographs)}

Good glossy original prints are required, black and white
\emphasize{only}; color plates cannot be reproduced.  Photographs cut
from other publications will not reproduce well, and usually infringe
copyright.  %Have the illustrations prepared so that when they are
reduced %they are 13.4cm (5.25.in) wide or less; alternatively, crop
the %illustration to the text width.  Supply both a large photograph
%and a reduced version pasted in the text.  The 
pasted-in version will
be used for size and placement only; a copy machine can be used for
this reduction.  If a reduction is not supplied, leave appropriate space
in text above the figure caption.  The publisher will take care of the
photographic reduction and mounting of the original glossy print in the
space provided by your pasted-in version.

\section{References}

\subsection{In the Text}

The reference system to be followed is the standard \apj\ system: author
name(s) followed by the year in parentheses, as in Abt (1990), or author
and year both in parentheses (Abt 1990).  Multiple authors would be
cited as (Groth \& Pebbles 1971) or (Kron, Hewitt, \& Wasserman 1984).
For more that three authors use ``et al.,'' e.g., (Press et~al.\ 1994).

\subsection{Reference List}
\label{reflist}

There is a {\tt references} environment that sets off
the list of references and adjusts spacing parameters.
\begin{quote}
\verb+\begin{references}+\\
\verb+\reference+ \arg{bibliographic information}\\
\verb+   .+\\
\verb+   .+\\
\verb+\end{references}+
\end{quote}
The \arg{bibliographic information} should be in the order directed by
Abt (1990): author, year, journal, volume, and page.  For instance,
the reference for this editorial would be typed in as
\begin{quote}
\verb+Abt, H. 1990, ApJ, 357, 1+
\end{quote}
Note that there is no comma following the author name(s), there is no
trailing period at the end of the reference, and the entire line is set
in the body typeface (no font changes).  See {\tt sample2.tex} for more
complex examples.

To refer to a paper from this conference, use the \verb+\adassxiii+ and
\verb+\paperref+ macros, and 2004 for the year.  For example,
\begin{quote}
\verb+\reference Zacharias, N.\ \& Zacharias, M.\ 2004,+\\
\verb+ \adassxiii, \paperref{O4-1}+
\end{quote}
will appear as
\begin{quote}
Zacharias, N.\ \& Zacharias, M.\ 2004, \adassxiii, \paperref{O4-1}
\end{quote}
in your preprint version of the paper; in the proceedings volume, the
``[O4-1]'' will be replaced with the actual page number of the paper.  

Care should be taken that each literature citation in the manuscript
has its counterpart in the reference list and vice versa.  Care should
also be given to checking the accuracy of the references---author(s),
date, volume, and page number.  The accuracy of the references is the
sole responsibility of the author.

\subsection{Abbreviations for Journals}

There are macros for many of the oft-referenced journals so that authors
may use the \LaTeX\ names rather than having to look up a particular
journal's specific abbreviation.  Any stylistic requirements of the
editors are taken care of by the macros, so authors need not be concerned
about such editorial preferences.

\begin{center}
\begin{tabular}{ll}
\verb+\aj+ & Astronomical Journal\\
\verb+\araa+ & Annual Review of Astronomy and Astrophysics\\
\verb+\apj+ & Astrophysical Journal\\
\verb+\apjl+ & \rule[.5ex]{2em}{.4pt}, Letters to the Editor\\
\verb+\apjs+ & \rule[.5ex]{2em}{.4pt}, Supplement Series\\
\verb+\ao+ & Applied Optics\\
\verb+\apss+ & Astrophysics and Space Science\\
\verb+\aap+ & Astronomy and Astrophysics\\
\verb+\aaps+ & \rule[.5ex]{2em}{.4pt}, Supplement Series\\
\verb+\azh+ & Astronomicheskii Zhurnal\\
\verb+\baas+ & Bulletin of the AAS\\
\verb+\jrasc+ & Journal of the RAS of Canada\\
\verb+\memras+ & Memoirs of the RAS\\
\verb+\mnras+ & Monthly Notices of the RAS\\
\verb+\pra+ & Physical Review A: General Physics\\
\verb+\prb+ & Physical Review B: Solid State\\
\verb+\prc+ & Physical Review C:\\
\verb+\prd+ & Physical Review D:\\
\verb+\prl+ & Physical Review Letters\\
\verb+\pasp+ & Publications of the ASP\\
\verb+\pasj+ & Publications of the ASJ\\
\verb+\qjras+ & Quarterly Journal of the RAS\\
\verb+\skytel+ & Sky and Telescope\\
\verb+\sovast+ & Soviet Astronomy\\
\verb+\ssr+ & Space Science Reviews\\
\verb+\zap+ & Zeitschrift f\"ur Astrophysik\\
\verb+\adassi+ & ADASS I (1991)\\
\verb+\adassii+ & ADASS II (1992)\\
\verb+\adassiii+ & ADASS III (1993)\\
\verb+\adassiv+ & ADASS IV (1994)\\
\verb+\adassv+ & ADASS V (1995)\\
\verb+\adassvi+ & ADASS VI (1996)\\
\verb+\adassvii+ & ADASS VII (1997)\\
\verb+\adassviii+ & ADASS VIII (1998)\\
\verb+\adassix+ & ADASS IX (1999)\\
\verb+\adassx+ & ADASS X (2000)\\
\verb+\adassxi+ & ADASS XI (2001)\\
\verb+\adassxii+ & ADASS XII (2002)\\
\verb+\adassxiii+ & ADASS XIII (2003)\\
\verb+\adassxiv+ & ADASS XIV (2004)\\
\verb+\adassxv+ & ADASS XV (2005)\\
\end{tabular}
\end{center}

\section{Examples}

These instructions give an overview of the basic markup commands that
need to be entered in a paper. Authors are encouraged to examine the
sample papers that are included with the style file; these examples
are named \verb+sample1.tex+ and \verb+sample2.tex+.  The file
\verb+sample1.tex+ is a paper prepared with the ADASSCONF macros
utilizing a \emphasize{minimal} amount of markup. A more ``complete''
paper requiring most of the capabilities of the package is provided as
\verb+sample2.tex+; this file is annotated with comments that describe 
the purpose of the markup.

\section{Reprints}

\edcomment{Since markup matches WGAS stuff, many choices exist.}
Reprints of papers for these Proceedings are not available.  Articles
may be copied from the published volume.

\section{Submission of Manuscripts}
\label{submit}

Completed manuscripts should submitted via anonymous FTP using the
following procedure:

\begin{enumerate}

\item Anonymous-FTP to \verb;xmm.vilspa.esa.es;.  Give
``anonymous'' as the login name and your email address as the password.

\item Change into the \verb+pub/adassxv+ directory:
\begin{quote}
\verb;cd pub/adassxv;
\end{quote}

\item Create a subdirectory whose name is the paper identification code:
\begin{quote}
\verb;mkdir P.205; (example for poster contribution P.205) 
\end{quote}

\item Change into that directory:
\begin{quote}
\verb;cd P.205; 
\end{quote}

\item Upload your files using the \quad\verb;put; command.  Identify the
\LaTeX\ manuscript giving it a name using the paper identification code
(e.g., ``P.205.tex'').  Any EPS files should be identified in a similar way
(e.g., ``P.205\_1.eps'', ``P.205\_2.eps''; see also \S 2.11). 
%Note that you
%will be able to see a listing of the files you upload using \quad\verb;ls;;
%however, you will not be able download, delete, or overwrite them after
%the upload. 
If you wish to upload any other files please name them
using the same convention, i.e., starting with your paper identifier
(e.g., P.205.junk).

\item Disconnect by typing \quad\verb;quit;.
\end{enumerate}

Here is how a sample session might look:
\begin{verbatim}
     % ftp xmm.vilspa.esa.es
     Name (emc2.relativity.com:aeinstein): anonymous  
     Password: A.Einstein@emc2.relativity.com
     ftp> cd pub/adassxv
     ftp> mkdir P.205
     ftp> cd P.205
     ftp> put P.205.tex
     ftp> put P.205_f1.eps
     ftp> quit
\end{verbatim}

If you have figures that cannot be sent in Encapsulated PostScript form, you can
mail them to the address given in 
%\S\ref{copyright}
\S 7
.  
%You may also
%optionally send a paper copy of your figure along with the figures to
%ensure proper placement of the figures within the manuscript.  

Manuscripts must be received no later than {\bf 31 October 2005}
in order to be assured publication in the Proceedings.  

\section{Copyright Agreement}
\label{copyright}

All authors should have returned a completed form of the Copyright
Assignment form, file during the conference. If you did not return the
signed form to us at the conference, you must download
\htmladdnormallinkfoot{the Copyright Assignment form}%
{http://adass.ipac.caltech.edu/docs/Copyright-Form-ASP.pdf},
and send or fax the completed form to the ADASS XV editors at:
\begin{quote}
     ADASS 2005 Proceedings\\
     c/o Carlos Gabriel \\
     European Space Astronomy Centre \\
     P.O.Box 50727\\
     E-28080 Madrid - Spain \\
     ~\\
     Fax: +34 91 8131 172
\end{quote}

\acknowledgements

The editors would like to acknowledge the other previous editors
for their work on this manual and the associated style files upon which
this manual and \verb+adassconf.sty+ are built.

\end{document}
